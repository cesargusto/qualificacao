%-----------------------------------------------------------------------------------------------------------------------------------------
% MODELO DE PROJETO DE QUALIFICA��O DE MESTRADO
%
% CENTRO FEDERAL DE EDUCA��O TECNOL�GICA DE MINAS GERAIS | CEFET-MG
% DEPARTAMENTO DE PESQUISA E P�S-GRADUA��O | DPPG
% AUTOR: LINHA DE PESQUISA EM MODELAGEM, APERFEI�OAMENTO E OTIMIZA��O DE PROCESSOS | MAOP
%
% PARTE: RESUMO
%
% ALUNO: Alexandre Frias Faria
%-----------------------------------------------------------------------------------------------------------------------------------------

\chapter*{\textbf{Resumo}}\label{Resumo}

\noindent
Este projeto de disserta��o de mestrado apresenta formula��o matem�tica e algoritmos para a vers�o de otimiza��o do Problema da k-Parti��o de N�meros. Este problema consiste em distribuir os elementos de um conjunto dado em $k$ subconjuntos disjuntos, de modo que as somas dos elementos de cada subconjunto fiquem no menor intervalo poss�vel. O primeiro m�todo consiste em aplicar a meta-heur�stica ILS (\emph{Iterated Local Search}) na solu��o gerada por um algoritmo aproximado. A estrat�gia � aplicar m�todo guloso em todas as fases do algoritmo, tanto na inicializa��o quanto para a escolha de movimentos do ILS. Outras propostas visam substituir o Or�culo do Algoritmo \ref{algoritmo1} pelo melhor resultado de um conjunto de heur�sticas. Por fim, o uso de algoritmos exatos para problemas relacionados, como o Problema da Soma de Subconjunto, s�o adaptados ao tema desse trabalho. Alguns resultados parciais j� publicados encontram-se no final do texto.

\vspace{0.5cm}



\noindent
\underline{PALAVRAS-CHAVE}: Problema da k-Parti��o de N�meros, Problema da Parti��o de N�meros, \emph{Iterated Local Search},  \emph{Complete Karmarkar-Karp Algorithm}.
\bigskip 